\chapter{Introduction}
\label{sec:introduction}

\section{Background and Motivation}
\begin{German}
    Die Bauindustrie ist einer der grössten Wirtschaftszweige, doch ihre Produktivität stagniert seit Jahrzehnten. In den zwei Jahrzehnten von 1995 bis 2015 wuchs ihre Produktivität nur um 1\%, weit unter dem globalen Wirtschaftsdurchschnitt von 2.8\%. Als einer der Gründe wird die geringe Automatisierung genannt \cite{barbosaReinventingConstructionRoute2017}.
\end{German}
\begin{English}
    The construction industry is one of the largest economic sectors, yet its productivity has stagnated for decades. In the two decades from 1995 to 2015, its productivity grew by only 1\%, far below the global economy's average of 2.8\%. One of the reasons cited is the low level of automation \cite{barbosaReinventingConstructionRoute2017}.
\end{English}

% Conventional Planning with CAD
\begin{German}
    Als einer der Gründe wird die geringe Automatisierung genannt. Bis zur Jahrtausendwende war die Bauplanung weitgehendst digitalisiert, insbesondere durch den Einsatz von Computer Aided Design (CAD) für die Erstellung von Bauzeichnungen. Es handelte sich um eine evolutionäre innovation, bei der die konventionelle planung optimiert wurde. Das manuelle Zeichnen am reissbrett wurde durch manuelles zeichnen am computer ersetzt, die einzelnen arbeitsschritte blieben weitgehnd dieselben. Bei der konventionelle planung wird ein reales objekt induktiv durch einzelne, zweidimensionale zeichnungen (z.B. grundriss, schnitt, detail) abgebildet. Die zeichnungen werden oft als isolierte dateien (z.B. dwg, dxf) gespeichert und sind dabei weder geometrisch noch semantisch miteinander verknüpft. änderungen am realen objekt müssen in allen zeichnungen manuell vorgenommen werden.
\end{German}

\begin{English}
    Until the turn of the millennium, construction planning had been largely digitized, primarily through the use of Computer-Aided Design (CAD) for creating construction drawings. This represented an evolutionary innovation, optimizing conventional planning methods. Manual drafting on the drawing board was replaced by manual drafting on the computer, while the individual work steps remained largely the same. In conventional planning, a real object is inductively represented by individual two-dimensional drawings (e.g., floor plan, section, detail). These drawings are often stored as isolated files (e.g., DWG, DXF) and are neither geometrically nor semantically linked. Any changes to the real object must be manually updated in all drawings.
\end{English}

% Model-Based Planning with BIM-Software
\begin{German}
    Ab der Jahrtausendwende begann sich Building Information Modelling (BIM) zu verbreiten. Es handelte sich um eine disruptive innovation, bei der die bisherigen arbeitsschritte grundlegend verändert wurden. Bei der modellbasierten planung wird ein dreidimensionales modell als digitales abbild des realen objekts erstellt. die zweidimensionalen zeichnungen werden anschliessend deduktiv aus dem modell abgeleitet. das modell wird als einheitliche datei (z.b. ifc) gespeichert und enthält sowohl geometrie als auch semantik. änderungen am modell werden automatisch in allen zeichnungen übernommen.

    Vorreiterrollen bei der Implementierung von BIM haben die skandinavischen Länder sowie grossbritanien. die implementierung der technologie gestaltete sich in der schweiz schwierig. Im Jahr 2021 wurde gerademal bei 20\% der Schweizer Bauunternehmen BIM eingesetzt (DE: 70\%, UK: 80\%). Die schweiz befindet sich damit im europäischen vergleich knapp im durchschnitt. \cite{heinrichSchweizImBIMEuropavergleich2022}. Unter anderem führte die fragmentierte Bauwirtschaft und der hohhe wettbewerbsdruck dazu, dass besonders kleinere Unternehmen die anfänglich hohen investitionskosten scheuten. \cite{ivanicErfolgreicheEinfuehrungBuilding2020}.
\end{German}

\begin{English}
    At the turn of the millennium, Building Information Modeling (BIM) began to gain traction. It represented a disruptive innovation that fundamentally changed previous workflows. In this context, BIM refers to both the technology (software) and the methodology (processes). In model-based planning, a three-dimensional model is created as a digital representation of the real object. The two-dimensional drawings are then deductively derived from the model. The model is stored as a unified file (e.g., IFC) and contains both geometry and semantics. Any changes made to the model are automatically reflected in all drawings.

    Scandinavian countries and the United Kingdom have taken leading roles in the implementation of BIM. However, integrating the technology in Switzerland has proven challenging. By 2021, only 20\% of Swiss construction companies had adopted BIM, compared to 70\% in Germany and 80\% in the UK. This places Switzerland slightly above the European average. \cite{heinrichSchweizImBIMEuropavergleich2022}. Among the factors contributing to this slow adoption are the fragmented nature of the construction industry and high competitive pressure, which led smaller companies, in particular, to shy away from the initially high investment costs. \cite{ivanicErfolgreicheEinfuehrungBuilding2020}. \
\end{English}

% Telecommunications Planning
\begin{German}
    Die Schweizer Telekommunikationsbranche hat das Potenzial der modellbasierten Planung erkannt. Während sich die traditionelle Bauwirtschaft typischerweise auf große, individuelle Bauvorhaben konzentriert, operiert die Telekommunikationsbranche mit einem hohen Volumen an kleineren, standardisierten Projekten, bei denen auf bestehenden Gebäuden Mobilfunkanlagen errichtet werden. Für die Planung solcher Vorhaben soll das jeweilige Gebäude effizient in ein digitales Modell überführt werden können.
\end{German}

\begin{English}
    The Swiss telecommunications industry has recognized the potential of model-based planning. While the traditional construction industry typically focuses on large, individual construction projects, the telecommunications sector operates with a high volume of smaller, standardized projects, involving the installation of mobile radio systems on existing buildings. For the planning of such projects, the respective building should be efficiently converted into a digital model.
\end{English}

% Scan-to-BIM vs Scan-to-GIS
\begin{German}
    Dafür haben sich unterschiedliche Ansätze entwickelt, bei dem Gebäude mittels Reality Capture in digitale Modelle überführt werden können:

    \begin{itemize}
        \item \textbf{Scan-to-GIS:} Bei Scan-to-GIS wird ein Oberflächenmodell für die Nutzung in Geoinformationssystemen (GIS) erzeugt. Dieses Modell besteht häufig aus tausenden kleinen Dreiecken (Meshes), die geometrisch nicht logisch gegliedert sind. Es enthält keine semantischen Informationen wie Objektklassen oder Bauteileigenschaften und dient primär der Visualisierung und geobezogenen Analyse. Während die Generierung relativ schnell erfolgt, ist eine nachträgliche Bearbeitung oder Strukturierung aufwändig.
        
        \textit{Analogie: Einscannen eines Textesdokuments geht schnell aber die nachträgliche Textbearbeitung ist schwierig.}

        \item \textbf{Scan-to-BIM:} Bei Scan-to-BIM wird ein BIM-Modell für die Nutzung in BIM-Software erzeugt. Dieses Modell besteht aus logisch gegliederten Bauteilen und enthält semantische Informationen. Es dient der detaillierten Planung und Dokumentation von Bauwerken. Die Generierung ist aufwändiger als bei Scan-to-GIS, da die Modelle eine höhere geometrische Genauigkeit und semantische Struktur aufweisen müssen. Für die Standortplanung müssen Baupläne erstellt werden können, was die Bearbeitbarkeit voraussetzt. 
        
        \textit{Analogie: Das Abtippen eines Textdokumentes in einem Textbearbeitungsprogramm dauert länger, dafür kann anschliessend effizient damit gearbeitet werden.} 
    \end{itemize}
\end{German}

\begin{English}
    Various approaches have been developed for converting buildings into digital models using reality capture:

    \begin{itemize}
        \item \textbf{Scan-to-GIS:} Scan-to-GIS involves creating a surface model for use in Geographic Information Systems (GIS). This model often consists of thousands of small triangles (meshes) that are not logically structured geometrically. It lacks semantic information such as object classes or building component properties and is primarily used for visualization and geospatial analysis. While the generation process is relatively quick, subsequent editing or structuring is time-consuming.
        
        \textit{Analogy: Scanning a text document is quick, but editing the text afterwards is difficult.}

        \item \textbf{Scan-to-BIM:} Scan-to-BIM involves creating a BIM model for use in BIM software. This model consists of logically structured building components and contains semantic information. It is used for detailed planning and documentation of buildings. Generation is more complex than with Scan-to-GIS, as the models must exhibit higher geometric accuracy and semantic structure. For site planning, building plans must be created, requiring editability.
        
        \textit{Analogy: Retyping a text document in a word processing program takes longer, but efficient work can be done afterwards.}
    \end{itemize}
\end{English}

\section{Research Objectives and Questions}
\begin{German}
    Sowohl BIM als auch Scan-to-BIM wurden in den letzten Jahren intensiv erforscht \cite{rochaSurveyScantoBIMPractices2021}. \\
    Die bestehende Literatur für BIM ist sehr umfangreich. Es konnten zahlreiche Fachbücher und wissenschaftliche Publikationen gefunden werden. Speziell für Scan-to-BIM konnten hingegen keine Bücher aus wissenschaftlichen Fachverlagen identifiziert werden. Das Thema wurde in BIM-Fachbüchern nur oberflächlich behandelt. Die Literatur besteht hier primär aus wissenschaftlichen Publikationen. Oft angetroffene Anwendungen liegen im klassischen Bausektor und der Dokumentation von historischen Bauten. Publikationen, die sich speziell mit der Standortplanung der Telekommunikationsbranche beschäftigten konnten keine gefunden werden. Die Anforderungen an die zu generierenden Modelle unterscheiden sich nach Anwendung. Im klassischen Bausektor und bei historischen Bauten werden häufig sowohl die Innen- als auch die Aussenstrukturen modelliert. Bei der Standortplanung beschränkt sich die Modellierung hauptsächlich auf Aussenstrukturen und ähnelt damit Scan-to-GIS-Verfahren. Der Fokus liegt auf automatisierten Verfahren. \\

    Für eine präzise Definition des Forschungsziel, sollen zunächst Ablaufsbegriffe definiert werden:
    
    \begin{itemize}
        \item \textbf{Workflow:} Ein Workflow ist eine Abfolge von Arbeitsschritten, die zur Erreichung eines Ziels durchgeführt werden. Auf strategischer Ebene soll technologieunabhängig geklärt werden, \textit{WAS}, \textit{WANN} und \textit{WARUM} gemacht wird.\\
        \textit{Beispiel: Segmentierung}

        \item \textbf{Framework:} Ein Framework ist die Abfolge von Methoden, die die Arbeitsschritte konkretisieren. Auf taktischer Ebene soll technologieunabhängig geklärt werden, \textit{WIE} die Arbeitsschritte durchgeführt werden. Methoden können dabei z.B. Algorithmen oder Verfahren sein.\\
        \textit{Beispiel: Segmentierung durch polyfit-Algorithmus}
        
        \item \textbf{Toolchain:} Die Toolchain ist die Abfolge von Tools, mit denen die Methoden implementiert werden. Auf operativer Ebene soll technologieabhängig geklärt werden, \textit{WOMIT} die Arbeitsschritte durchgeführt werden. Tools können z. B. Software, Skripte oder Plattformen sein.\\
        \textit{Beispiel: Segmentierung durch in Python implementierter polyfit-Algorithmus}
    \end{itemize}

% Forschung durch Entwerfen = Reasearch by Design

    In einem ersten Schritt soll mittels der Methodik "Forschung durch Entwerfen" ein auf die Standortplanung zugeschnittenes Framework entwickelt werden. Dafür sollen bestehende Workflows und Methoden aus der Forschung herangezogen werden. In einem zweiten Schritt soll für das Framework eine operativ ausführbare Toolchain entwickelt und dieses dann implementiert werden.

% Hauptfrage: Was will ich wissen?
% Unterfrage: Wie finde ich es heraus?

    \begin{itemize}
        \item \textbf{Hauptfrage:} Wie kann der Scan-to-BIM-Workflow für die Standortplanung in der Telekommunikationsbranche automatisiert werden?
        \begin{itemize}
            \item \textbf{Unterfrage 1:} Welche Methoden sind für die Entwicklung eines automatisierten Scan-to-BIM-Frameworks für die Standortplanung geeignet?
            \item \textbf{Unterfrage 2:} Welche Tools sind für die Entwicklung einer automatisierten Scan-to-BIM-Toolchain für die Standortplanung geeignet?
        \end{itemize}
    \end{itemize}
\end{German}

\begin{English}
    Both BIM and Scan-to-BIM have been intensively researched in recent years \cite{rochaSurveyScantoBIMPractices2021}. \\
    The existing literature on BIM is extensive, with numerous textbooks and scientific publications available. However, no books from scientific publishers could be identified specifically for Scan-to-BIM. The topic has only been superficially addressed in BIM textbooks, with the literature primarily consisting of scientific publications. Common applications are found in the traditional construction sector and the documentation of historical buildings. No publications specifically addressing the site planning of the telecommunications industry could be found. The requirements for the generated models vary depending on the application. In the traditional construction sector and historical buildings, both interior and exterior structures are often modeled. In site planning, modeling is mainly limited to exterior structures. Historical buildings often feature geometrically complex shapes, requiring high geometric accuracy for capture. In site planning, the level of generalization is higher, focusing instead on efficient processes. The requirements profile is thus closer to a Scan-to-GIS process than the publications covered in the literature. \\

    To precisely define the research objective, the necessary terms must first be defined:

    \begin{itemize}
        \item \textbf{Workflow:} A workflow is a sequence of work steps performed to achieve a goal. At the strategic level, it clarifies technology-independent \textit{WHAT}, \textit{WHEN}, and \textit{WHY} is done.\\
        \textit{Example: Segmentation}

        \item \textbf{Framework:} A framework is the sequence of methods that concretize the work steps. At the tactical level, it clarifies technology-independent \textit{HOW} the work steps are performed. Methods can include algorithms or procedures.\\
        \textit{Example: Segmentation by polyfit algorithm}

        \item \textbf{Toolchain:} The toolchain is the sequence of tools used to implement the methods. At the operational level, it clarifies technology-dependent \textit{WITH WHAT} the work steps are performed. Tools can include software, scripts, or platforms.\\
        \textit{Example: Segmentation by polyfit algorithm implemented in Python}
    \end{itemize}

    The first step involves developing a site planning-specific framework. Existing workflows and methods from research will be used for this purpose. The second step involves developing an operationally executable toolchain for the framework and implementing it. Existing tools from practice will be used for this purpose. This leads to a main question with two sub-questions:

    \begin{itemize}
        \item \textbf{Main Question:} How can an application-oriented framework for implementing a Scan-to-BIM workflow for site planning in the telecommunications industry be developed and practically applied?
        \begin{itemize}
            \item \textbf{Sub-Question 1:} What workflows and methodological approaches are described in the current literature, and how can they be systematically categorized?
            \item \textbf{Sub-Question 2:} How can an individual framework with a toolchain be developed and implemented based on this information?
        \end{itemize}
    \end{itemize}
\end{English}

\section{Thesis Structure}
\begin{German}
    Für die Beantwortung der Fragestellungen wird die Arbeit in zwei Teile untergliedert. Dabei widmet sich jeder Teil der Beantwortung einer Fragestellung:

    \begin{itemize}
        \item \textbf{Teil 1:} Im ersten Teil soll mit einer vergleichenden Analyse die erste Fragestellung beantwortet werden. In Kapitel 2 wird zunächst eine Einführung in die drei Themenblöcke Mobilfunk, BIM und Scan-to-BIM gegeben. Es soll damit eine Basis für den zweiten Teil geschaffen werden. Im Kaptiel 3 erfolgt die eigentliche Analyse. Einige der in Kapitel 2 vorgestellten Scan-to-BIM-Technologien werden genauer untersucht und auf ihre Eignung beurteilt. Auf dieser Grundlage wird ein konkretes Framework entwickelt.
        \item \textbf{Teil 2:} Im zweiten Teil soll das entwickelte Framework auf ein praxisnahes Fallbeispiel angewendet werden. Dafür wird im Kapitel 4 zunächst das Ausgangsproblem identifiziert. Im Kapitel 5 wird anschliessend auf die Resultate der Implementierung eingegangen. Abschliessend werden im Kapitel 6 die Ergebnisse zusammengefasst und ein Ausblick auf zukünftige Forschung gegeben.
    \end{itemize} 
\end{German}

\begin{English}
    To address the research questions, this thesis is divided into two parts, with each part focusing on one of the research questions:

    \begin{itemize}
        \item \textbf{Part 1:} The first part aims to answer the first research question through a comparative analysis. Chapter 2 provides an introduction to the three main topics of telecom, BIM, and Scan-to-BIM, laying the foundation for the subsequent analysis. Chapter 3 delves into the analysis itself, examining several Scan-to-BIM technologies introduced in Chapter 2 and evaluating their suitability. Based on this analysis, a concrete framework is developed.
        \item \textbf{Part 2:} The second part involves applying the developed framework to a practical case study. Chapter 4 identifies the initial problem, while Chapter 5 discusses the results of the implementation. Finally, Chapter 6 summarizes the findings and provides an outlook on future research.
    \end{itemize}
\end{English}