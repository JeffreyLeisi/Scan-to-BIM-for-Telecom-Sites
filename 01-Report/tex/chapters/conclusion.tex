\chapter{Conclusion}
\label{sec:conclusion}

% The conclusion summarizes the key findings, reflects on the research contributions,
% and provides final thoughts on the study. It should also outline potential future work.

\section{Introduction}
TODO: Briefly introduce the purpose of the conclusion.
% - Explain that this chapter summarizes the main findings.
% - Highlight the broader significance of the research.
% - Mention that potential future work will also be discussed.

\section{Summary of Key Findings}
TODO: Recap the most important results of the study.
% - What were the main research questions, and how were they answered?
% - Summarize the key takeaways from the results and discussion.
% - Avoid introducing new information—only restate what has been covered.

\section{Contributions of the Research}
TODO: Explain the impact of the research.
% - What are the main contributions to the field of Scan-to-BIM and telecom site planning?
% - How does this research advance knowledge or industry practices?
% - Mention any methodological, theoretical, or practical contributions.

\section{Limitations and Challenges}
TODO: Recap the main limitations of the study.
% - What were the primary constraints (e.g., data availability, computational resources)?
% - How might these limitations have influenced the results?
% - Acknowledge any potential biases or assumptions made.

\section{Recommendations for Future Work}
TODO: Suggest directions for further research.
% - What aspects were not covered in this study that should be explored further?
% - Are there potential improvements to the methodology or technology?
% - How can this research be expanded to benefit industry adoption?

\section{Final Remarks}
TODO: End with a strong closing statement.
% - Reflect on the significance of the research as a whole.
% - Provide a concluding thought on the future of Scan-to-BIM automation.
% - Keep it concise and impactful.

% General Advice:
% - Keep the conclusion concise—do not introduce new results or analyses.
% - Clearly state how the research has contributed to the field.
% - Provide a balanced view, acknowledging both achievements and challenges.
% - Leave the reader with a strong and forward-looking closing statement.
