\chapter{Conclusion}
\label{sec:conclusion}

\begin{German}
    Im Rahmen dieser Arbeit wurde ein Ablauf entwickelt, der die Generierung eines BIM-Modells aus einer Punktwolke ermöglicht. Auf Grundlage der Design Science Research Methodology (DSRM) wurde ein entsprechendes Framework konzipiert, als Pipeline implementiert und im Rahmen einer Fallstudie in Zusammenarbeit mit Axians demonstriert. Die Fallstudie zeigte exemplarisch, dass das Framework grundsätzlich in der Lage ist, ein BIM-Modell zu erzeugen. Damit konnte nicht nur beantwortet werden, ob eine Automatisierung möglich ist, sondern auch, wie diese konkret realisiert werden kann.

    Es wurde ein fünfstufiger Ablauf vorgeschlagen, bestehend aus \textit{Datenakquise (1)}, \textit{Preprocessing (2)}, \textit{Geometrischer Segmentierung (3)}, \textit{Semantischer Segmentierung (4)} und \textit{Modellierung (5)}. Mit Ausnahme der Datenakquise sowie Teilen der Modellierung konnten die Schritte erfolgreich automatisiert werden. Zum Einsatz kamen dabei unter anderem \textit{CloudCompare}, \textit{Python}, \textit{Revit} und \textit{Dynamo}.

    Im Verlauf der Arbeit wurden mehrere Herausforderungen identifiziert, welche die praktische Anwendbarkeit derzeit einschränken. Diese lassen sich den drei Kategorien \textit{Geometrische Rekonstruktion (1)}, \textit{Automatisierung (2)} sowie \textit{Usability und Integration (3)} zuordnen. Für jede dieser Kategorien wurden gezielte Lösungsansätze formuliert. Das Prjekt war ein erster Versuch, die Automatisierung der Scan-to-BIM-Prozesskette zu realisieren. Sollte die geometrische Rekonstruktion künftig weiter verbessert werden, erscheint ein produktiver Einsatz des Frameworks in der Standortplanung realistisch. Damit würde ein Beitrag zur Effizienzsteigerung in der Bauindustrie geleistet.
\end{German}

\begin{English}
    In this thesis, a process was developed that enables the generation of a BIM model from a point cloud. Based on the Design Science Research Methodology (DSRM), a corresponding framework was designed, implemented as a pipeline, and demonstrated in a case study in collaboration with Axians. The case study exemplified that the framework is fundamentally capable of producing a BIM model. This not only answered whether automation is possible but also how it can be concretely realized.

    A five-step process was proposed, consisting of \textit{Data Acquisition (1)}, \textit{Preprocessing (2)}, \textit{Geometric Segmentation (3)}, \textit{Semantic Segmentation (4)}, and \textit{Modeling (5)}. With the exception of data acquisition and parts of modeling, the steps were successfully automated. Tools such as \textit{CloudCompare}, \textit{Python}, \textit{Revit}, and \textit{Dynamo} were employed.

    Throughout the work, several challenges were identified that currently limit practical applicability. These can be categorized into three areas: \textit{Geometric Reconstruction (1)}, \textit{Automation (2)}, and \textit{Usability and Integration (3)}. Targeted solutions were formulated for each category. If geometric reconstruction continues to improve, the productive use of the framework in site planning for telecommunications equipment appears realistic. This would contribute to increased efficiency in the construction industry.
\end{English}

\subsubsection{Data Availability Statement}
\begin{German}
    Der Quellcode dieser Arbeit wird ab Ende Juli 2025 auf GitHub veröffentlicht. Eine technische Dokumentation wird nicht bereitsgestellt. Die Punktwolke, die als Grundlage für die Fallstudie diente, wurde von Axians zur Verfügung gestellt und wird daher nicht öffentlich zugänglich sein. Fragen können gerne per E-Mail an leisij@student.ethz.ch gestellt werden. 
\end{German}

\begin{English}
    The source code of this thesis will be published on GitHub at the end of July 2025. A technical documentation will not be provided. The point cloud used as the basis for the case study was provided by Axians and will therefore not be publicly accessible. Questions can be sent via email to leisij@student.ethz.ch. 
\end{English}

\subsubsection{Conflicts of Interest}
\begin{German}
    Die Zusammenarbeit mit der Axians erfolgte im Rahmen eines betreuten Praxisprojekts. Es bestand kein finanzieller Ausgleich oder vertraglicher Einfluss auf die Inhalte oder Ergebnisse dieser Arbeit.
\end{German}

\begin{English}
    The collaboration with Axians was part of a supervised practical project. There was no financial compensation or contractual influence on the content or results of this work.
\end{English}