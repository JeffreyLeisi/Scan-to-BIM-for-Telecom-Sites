\chapter*{Abstract}
\addcontentsline{toc}{chapter}{Abstract}

\begin{itemize}
    \item \textbf{Introduction to the Topic}  
    % Briefly introduce the field of research and the problem being addressed.
    % Example: "The efficient planning of telecommunications sites requires accurate 3D models. Traditionally, these models are created manually, leading to inefficiencies."

    \item \textbf{Research Objective}  
    % Clearly define the aim of the research. What is the specific problem you are solving?  
    % Example: "This thesis explores the automation of Scan-to-BIM workflows to enhance efficiency in telecom site planning."

    \item \textbf{Methodology}  
    % Briefly outline the approach taken to achieve the research goal.
    % Mention key steps such as:
    % - Data acquisition (e.g., point cloud processing)
    % - Algorithmic processing (segmentation, meshing, feature extraction)
    % - Evaluation methods (comparison of different algorithms)

    \item \textbf{Results}  
    % Summarize the key findings of the research. Keep it concise but include concrete outcomes.  
    % Example: "The results show that Algorithm X reduces processing time by Y% while maintaining an accuracy of Z%."

    \item \textbf{Conclusion and Impact}  
    % State the broader implications of the research.  
    % How does this work contribute to the field? What are potential future directions?  
    % Example: "This study demonstrates the feasibility of automated Scan-to-BIM workflows, contributing to the digital transformation of telecom site planning."
\end{itemize}

% General advice:
% - Keep it between 150-250 words.
% - Avoid citations, figures, and overly technical jargon.
% - Make it understandable to a broad audience, not just domain experts.

\section*{Keywords}
BIM, Scan-to-BIM, telecommunications, automation, point cloud processing