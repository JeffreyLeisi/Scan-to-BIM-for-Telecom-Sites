\chapter*{Abstract}
\addcontentsline{toc}{chapter}{Abstract}

\begin{German}
    Die vorliegende Bachelorarbeit untersucht die Automatisierbarkeit von Scan-to-BIM-Prozessen für die Telecom Standortplanung. Ziel war die Entwicklung eines automatisierten Frameworks zur Generierung eines BIM-Modells der Gebäudeaussenstruktur ausgehend von einer Punktwolke. Das Framework wurde methodisch nach der Design Science Research Methodology (DSRM) entwickelt. Darauf aufbauend wurde eine Pipeline implementiert und im Rahmen einer Fallstudie validiert. Zum Einsatz kamen die Software CloudCompare und Autodesk Revit. Für die Automatisierung wurden die Programmiersprache Python sowie die visuelle Programmierschnittstelle Dynamo verwendet.

    Die Ergebnisse der Fallstudie zeigen, dass eine weitgehend automatisierte Umwandlung von Punktwolken in BIM-Modelle auf LOD-200-Niveau grundsätzlich möglich ist. Für eine praxisgerechte Anwendung waren jedoch zusätzliche manuelle Anpassungen erforderlich. Darüber hinaus bestehen Herausforderungen hinsichtlich der geometrischen Genauigkeit und Modellvollständigkeit. Zur adressierung dieser Einschränkungen wurden mehrere Lösungsansätze entwickelt. Gelingt deren Umsetzung, erscheint ein produktiver Einsatz des Frameworks in der Standortplanung realistisch.
\end{German}

\begin{English}
    This bachelor's thesis investigates the automation of Scan-to-BIM processes for telecommunications site planning. The goal was to develop an automated framework for generating a BIM model of the building's exterior structure from a point cloud. The framework was methodically developed following the Design Science Research Methodology (DSRM). Based on this, a pipeline was implemented and validated in a case study. The software CloudCompare and Autodesk Revit were used. For automation, the programming language Python and the visual scripting interface Dynamo were used.

    The results of the case study demonstrate that a largely automated conversion of point clouds into BIM models at LOD-200 level is fundamentally possible. However, additional manual adjustments were required for practical application. Furthermore, challenges regarding geometric accuracy and model completeness exist. Several solutions have been developed to address these limitations. If successfully implemented, the productive use of the framework in site planning appears realistic.
\end{English}

\section*{Keywords}
Scan-to-BIM, Telecommunications Site Planning, Point Cloud Processing, PolyFit, ifcopenshell, Autodesk Revit, Dynamo, Python, Design Science Research Methodology (DSRM)